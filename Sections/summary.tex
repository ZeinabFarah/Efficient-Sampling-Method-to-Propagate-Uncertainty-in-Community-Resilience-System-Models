    Uncertainty is inherent in many modeling and experimental processes in science and engineering \cite{dror_modeling_2002}. Uncertainty propagation is an integral part of risk and resilience assessment and can impact practical decision-making in engineering applications \cite{adhikari_experimental_2009,alpak_techniques_2013}. While there are many available techniques for uncertainty propagation, Monte Carlo (MC) simulation methods are the most widely used to deal with uncertainties in various fields. MC methods are particularly desirable since they are simple to construct and maintain computational tractability. However, in some cases dealing with rare events such as natural hazards, the sampling error of MC methods can greatly affect the results. Sampling error can be reduced by increasing the sample size, thereby increasing the computational time. In this situation, the MC simulation method and a variance reduction technique should be used together to reduce the sampling error without significantly increasing the sample size.\\
    In engineering practices, mathematical models are used to depict natural phenomena using various assumptions and parameters. These models cannot accurately represent all features of the system under study. This is because: (\romannumeral 1) some events that occur during system operation are inherently random, and (\romannumeral 2) some phenomena are still poorly understood (e.g., due to a lack of experimental results). As a result, there is uncertainty regarding both the values of the model's input parameters and variables (known as parameter uncertainty) and the theories underpinning the structure of the model (known as model uncertainty). \\
    While many sources of uncertainty may exist, they are generally categorized as either aleatory or epistemic: \textit{Aleatory uncertainty} is related to the intrinsically random nature of several phenomena occurring during system operation. \textit{Epistemic uncertainty} is associated with the lack of knowledge about some properties and conditions of the phenomena underlying the behavior of the systems. This uncertainty manifests itself in the model representation of the system behavior in terms of both (model) uncertainty in the hypotheses assumed and (parameter) uncertainty in the (fixed but poorly known) values of the internal parameters of the model. It should be noted that while epistemic uncertainty can be reduced by gathering information and data to improve knowledge of the system's behavior, aleatory uncertainty cannot. \\
    In general, several available techniques for uncertainty propagation can be classified into analytical methods, graphical methods, probabilistic methods, and non-probabilistic methods \cite{hayes_uncertainty_2011}.   
    
    Probabilistic methods use probabilistic distributions to describe uncertainties in different aspects, such as events, demand, damage state, recovery, and decision-making. One of the most well-known probabilistic methods is called Monte Carlo Simulation (MCS). The Monte Carlo method has numerous types and is diverse. Still, all of its algorithms rely on randomly generating samples from a defined input domain, such as a set of probability density functions, to solve problems that do not have an analytical solution. A set of random values for all uncertain input parameters forms an input scenario or a realization, and the model outputs are estimated for each input scenario.\\

    